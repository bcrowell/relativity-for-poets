\chapter{Motion is relative}

\section{Galileo versus Aristotle}

Once upon a time, there was an ornery man who liked to argue.
He was undiplomatic and had a talent for converting allies into enemies.
His name was Galileo, and it was his singular misfortune to be correct in most of his opinions.
When he was arguing, Galileo had a few annoying habits. One was to answer a perfectly sound theoretical
argument with a contradictory experiment or observation. Another was that, since he
knew he was right, he freely made up descriptions of experiments that he hadn't actually done.

Galileo's true opponent was a dead man, the ancient Greek philosopher Aristotle.
Aristotle, probably based on generalizations from everyday experience, had come up
with some seemingly common-sense theories about motion.

\pagebreak

The figure shows an example of the kind of observation that might lead you to
the same conclusions as Aristotle. It shows a series of snapshots in the motion
of a rolling ball. Time moves forward as we go up the page.
Because the ball slows down and eventually stops, it traces out the shape of a
letter ``J.'' If you get rid of the artistic details, and connect the drawings of
the ball with a smooth curve, you get a graph, with position and time on the
horizontal and vertical axes. (If we had wanted to, we could have interchanged
the axes or reversed either or both of them. Making the time axis vertical, and
making the top point toward the future, is a standard convention in relativity,
like the conventional orientation of the compass directions on a map.)

Once the ball has stopped completely, the graph becomes a vertical line. Time
continues to flow, but the ball's position is no longer changing. According to
Aristotle, this is the natural behavior of any material object. The ball may move
because someone kicks it, but once the force stops, the motion goes away.
In other words, vertical lines on these graphs are special. They represent a natural, universal
state of rest.

\begin{figure}
\fignakedfullpage{soccer-ball}
\end{figure}

Galileo adduced two main arguments against the Aristotelian view. First, he said
that there was a type of force, friction, which was the reason that things slowed down.
The grass makes a frictional force on the ball. If we let the grass grow too high,
this frictional force gets bigger, and the ball decelerates more quickly. If we mow
the grass, the opposite happens. Galilo did experiments in which he rolled a smooth
brass ball on a smooth ramp, in order to make the frictional force as small as possible.
He carried out careful quantitative observations of the balls' motion, with time
measured using a primitive water clock. By taking measurements under a variety of
conditions, he showed by extrapolation from his data that if the ramp was perfectly level,
and friction completely absent, the ball would roll forever. 

Today we have easier ways
to convince ourselves of the same conclusion. You've seen a puck glide frictionlessly on an air hockey table
without slowing down. Or the next time you're on an airplane, traveling at nearly the speed of sound,
try dropping a coin in the aisle. It continues to move along with the plane; Aristotle would
have predicted that once it was no longer being encouraged by your hand to move forward, it
would come to a stop (relative to the ground) --- becoming a deadly projectile and probably killing
some passengers!

Galileo's second argument was based on astronomical observations. Of course even the ancients
had known that the sun, moon, stars, and planets rose in the east and set in the west. These bodies
didn't seem to naturally slow down, but they did go in circles around the earth, and that
could in fact be seen as reinforcing Aristotle's theory. After all, when we measure the motion
of an object, such as the rolling ball, we need some point of reference, like the dandelion on
the left, with which to compare. (Aboard an airplane, you subconsciously switch your frame of
reference to that of the cabin.) Aristotle's theory can be taken as a statement that there is
special or ``preferred'' frame of reference, which is the one attached to the earth. The earth
is special, Aristotle insists. We should always measure motion relative to it. And since the earth is
so special, it makes sense that the heavens revolve around it.

But in 1610, Galileo looked at the planet Jupiter through a telescope that he had invented, and
saw something startling. The planet, which looks like a bright and starlike point to the naked eye,
appeared as a disk, and accompanying it were four points of light. Observing them from one
night to the next, as shown in the notebook page reproduced here, he saw that they wove their
way back and forth, ahead of and behind Jupiter's disk. He inferred that they were moons
that were circling Jupiter, and that he was seeing the circles from the side, so
that they appeared flat. The analogy with our diagram of the rolling ball is nearly exact,
except that Galileo worked his way down the page from night to night, rather than going up.

The behavior of these moons is nothing like what Aristotle would have predicted. The earth
plays no special role here. The moons circle Jupiter, not the earth.

\begin{figure}
\fignakedfullpage{galileo-jovian-moons-sketch}
\end{figure}

\pagebreak

\section{The law of inertia}

We would like to summarize these observations in the form of some overarching law or
principle. The general idea is that motion doesn't naturally stop. It naturally
\emph{continues}, at constant speed and in a straight line.
Closely linked to this is the fact that no frame of reference is
preferred. But putting these ideas into precise language is a little tricky. Among the people who tried have been
Galileo (1564-1642), his successor Isaac Newton (1642-1726), and
Albert Einstein (1879-1955), and none of them really got it quite right by modern standards.

The main problem is in pinning down what we mean by saying that motion ``naturally'' continues. We have in mind
a description of what an object does when no forces act on it. But how do we know whether this is the case?
The first figure on the facing page shows an overhead view of a person whirling a rock in a circle on the end
of a string. We have a number of cues to tell us  that there is a force acting on the rock.
The string is taut, it's tied around the rock, and so on. If the string breaks, the force stops acting,
and the rock flies off in a straight line.

\begin{figure}
\fignakedfullpage{rock-on-string-and-earth}
\end{figure}

But what if the object going in a circle is the earth orbiting the sun? We've been told in school
that the force acting on it is the sun's gravity, but what signs do we have that this force is acting?
Unlike a hypothetical bug riding on the circling rock, we riders on the earth get no physical sensation
to tell us that this force exists at all.
If it stopped acting, we wouldn't feel any immediate sign of its cessation.
As suggested by the figure, we would just eventually notice that we seemed to be receding from the sun.

The reason we don't sense the sun's gravitational force is that it acts simultaneously and with
equal effect on both the earth and our bodies. Perhaps a simpler example is an amusement park ride near
where I live, called the Supreme Scream. Riders are lifted in chairs up to the top of a tower and then
dropped in free-fall straight down toward the ground. A popular thing to do on this ride is to get a coin ready at the
top, and then release it once the free fall has begun. It  hangs in the air in front of you, apparently
weightless, although both you and it are actually accelerating downward at the same rate. Once you get
off the ride, the change you feel is not any sudden reappearance of the earth's gravity, which was present all
along. It's the resumption of the usual force of the ground pressing up on your feet.

In this sense, gravity is different from the other forces of nature, such as magnetic or nuclear forces.
We can't shield against it, and we can't necessarily tell whether it's even acting. This is an example
of something called the \emph{equivalence principle}, which we'll come back to later in our study of relativity.

\begin{figure}
\fignakedfullpage{waage-box}
\end{figure}

For these reasons, we state a \emph{law of inertia} in a way that sidesteps the question of gravity.
We imagine building a box with layers of shielding, as suggested in the cutaway view in the fanciful figure,
where the box is floating in outer space.
One layer of shielding is aluminum foil to keep out any external electric forces. Another layer provides
shielding against magnetism. (There are special products sold for this purpose, with trade names such
as mu-metal.) We shield against all the forces except for gravity, which, as far as we know, cannot be
shielded against. Now inside the box we introduce a black ball and a white ball. The roles played by the
two balls are completely symmetrical, but if we like, we can think of one ball as playing the role of
an observer. For this reason I've drawn a ladybug clinging to the white ball. The bug can, if she
likes, define herself to be at rest, so that in her frame of reference only the black ball is moving.

The law of inertia states that if all nongravitational forces are prevented from acting on the balls,
then each will move in a straight line, at constant speed, relative to the other ball. This type of
straight-line, constant-speed motion is called inertial motion. To the ladybug,
the black ball's motion appears inertial.

Although I drew the box drifting in the interstellar void, nothing goes wrong if we apply this definition
with the box resting on a tabletop on earth. In this case, the two balls will be affected equally by the earth's
gravity, and each will still move inertially relative to the other.
Nor is such a box completely impractical to build;
Harold Waage at Princeton has made something similar to it as a lecture demonstration.

If you're accustomed to dictionary-style definitions, or the kind of definitions used in mathematics,
it may seem strange to define something using a piece of apparatus.
This is in fact a very common and philosophically well justified approach called operationalism.
The idea is that if we want to define something,
we should simply spell out the operations needed in order to measure or verify it.
For example, the operational definition of time would be that time is what is measured by a clock.

Implicit in this description is the fact that no frame of reference is special. We could have perched the
bug on the black ball, and that would have been an equally valid frame.
