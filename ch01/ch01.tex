\chapter{Galilean relativity}\label{ch:galilean-relativity}

\section{Galileo versus Aristotle}

Once upon a time, there was an ornery man who liked to argue.
He was undiplomatic and had a talent for converting allies into enemies.
His name was Galileo, and it was his singular misfortune to be correct in most of his opinions.
When he was arguing, Galileo had a few annoying habits. One was to answer a perfectly sound theoretical
argument with a contradictory experiment or observation. Another was that, since he
knew he was right, he freely made up descriptions of experiments that he hadn't actually done.

Galileo's true opponent was a dead man, the ancient Greek philosopher Aristotle.
Aristotle, probably based on generalizations from everyday experience, had come up
with some seemingly common-sense theories about motion.

\pagebreak

The figure \vpageref{fig:soccer-ball} shows an example of the kind of observation that might lead you to
the same conclusions as Aristotle. It shows a series of snapshots in the motion
of a rolling ball. Time moves forward as we go up the page.
Because the ball slows down and eventually stops, it traces out the shape of a
letter ``J.'' If you get rid of the artistic details, and connect the drawings of
the ball with a smooth curve, you get a graph, with position and time on the
horizontal and vertical axes. (If we had wanted to, we could have interchanged
the axes or reversed either or both of them. Making the time axis vertical, and
making the top point toward the future, is a standard convention in relativity,
like the conventional orientation of the compass directions on a map.)

Once the ball has stopped completely, the graph becomes a vertical line. Time
continues to flow, but the ball's position is no longer changing. According to
Aristotle, this is the natural behavior of any material object. The ball may move
because someone kicks it, but once the force stops, the motion goes away.
In other words, vertical lines on these graphs are special. They represent a natural, universal
state of rest.

\begin{figure}\label{fig:soccer-ball}
\fignakedfullpage{soccer-ball}
\end{figure}

Galileo adduced two main arguments against the Aristotelian view. First, he said
that there was a type of force, friction, which was the reason that things slowed down.
The grass makes a frictional force on the ball. If we let the grass grow too high,
this frictional force gets bigger, and the ball decelerates more quickly. If we mow
the grass, the opposite happens. Galilo did experiments in which he rolled a smooth
brass ball on a smooth ramp, in order to make the frictional force as small as possible.
He carried out careful quantitative observations of the balls' motion, with time
measured using a primitive water clock. By taking measurements under a variety of
conditions, he showed by extrapolation from his data that if the ramp was perfectly level,
and friction completely absent, the ball would roll forever. 

Today we have easier ways
to convince ourselves of the same conclusion. For example, you've seen a puck glide frictionlessly
across an air hockey table without slowing down.
Sometimes Galileo's analysis may be hard to accept. Running takes us a lot of effort, and
it seems as though we need to apply a continuous force to keep going. The figure above shows what's really
happening. At b, the runner's heel strikes the ground, and friction slows him down. A forward
force in c serves only to recover from the loss in speed.

\begin{figure}\label{fig:runner}
\fignakedfullpage{runner}
\end{figure}

Galileo's second argument had to do with the fact that we can't judge motion except by
comparing with some reference point, which we consider to be stationary. Aristotle
used the earth as a reference point. 
But in 1610, Galileo looked at the planet Jupiter through a telescope that he had invented, and
saw something startling. The planet, which looks like a bright and starlike point to the naked eye,
appeared as a disk, and accompanying it were four points of light. Observing them from one
night to the next, as shown in the notebook page reproduced here, he saw that they wove their
way back and forth, to the east and west of Jupiter's disk, in the plane of the solar system.
He inferred that they were moons
that were circling Jupiter, and that he was seeing the circles from the side, so
that they appeared flat. The figure \vpageref{fig:galileo-jovian-moons-sketch} shows the original diagrams from Galileo's notebook.
(The analogy with our diagram of the rolling ball is nearly exact,
except that Galileo worked his way down the page from night to night, rather than going up.)

The behavior of these moons is nothing like what Aristotle would have predicted. The earth
plays no special role here. The moons circle Jupiter, not the earth. Galileo's observations demoted
the earth from its Aristotelian place as the universal reference point for the rest of the cosmos,
making it into just one of several planets in the solar system.

\begin{figure}\label{fig:galileo-jovian-moons-sketch}
\fignakedfullpage{galileo-jovian-moons-sketch}
\end{figure}

\vfill\pagebreak

\section{Frames of reference}

If Aristotle had grown up on Jupiter, maybe he would have considered Jupiter to be the natural reference
point that was obviously at rest. In fact there is no one reference point that is always right. Psychologically,
we have a strong tendency to think of our immediate surroundings as being at rest. For example, when you fly
on an airplane, you tend to forget that the cabin is hurtling through the sky at nearly the speed of sound.
You adopt the \emph{frame of reference} of the cabin.
If you were to describe the motion of the drink cart progressing down the aisle with your nuts and soda,
you might say, ``Five minutes ago, it was at row 23, and now it's at row 38.'' You've implicitly laid out
a number line along the length of the plane, with a reference point near the front hatch (where ``row zero'' would be).
The measurements like 23 and 38 are called \emph{coordinates}, and in order to define them you need to pick
a frame of reference.

\begin{figure}\label{fig:cow-and-car}
\fignakedfullpage{cow-and-car}
\end{figure}

In the top part of the figure \vpageref{fig:cow-and-car}, we adopt the cow's frame of reference. The cow sees itself as being at rest,
while the car drives by on the road. The version at the bottom shows the same situation in
the frame of the driver,
who considers herself to be at rest while the scenery rolls by. To the driver, it's the cow, the grass, and
the trees that are moving.

Aristotle believed that there was one special, or \emph{preferred} frame of reference, which was
the one attached to the earth. A basic principle of relativity is that there is no such
preferred frame.

\vfill\pagebreak

\section{Inertial and noninertial frames}

That's not to say that all frames of reference are equally good. It's a little like the bitter joke
in George Orwell's satirical novel Animal Farm: ``All animals are equal, but some animals are more equal than others.''
For an example of the distinction between good and bad frames of reference, consider a low-budget movie in which
the director wants to depict an earthquake.
The easy way to do it is to shake the camera around --- but we can tell it's fake,
e.g., because although the actors can pretend to fall down, chairs don't topple and books don't fall off of shelves.
The shaking camera is not a valid frame.

\begin{figure}[h]\label{fig:plane-noninertial}
\fignakedfullpage{plane-noninertial}
\end{figure}

Similarly, when a plane hits bad turbulence, it's possible for passengers to go popping out of their seats and hit their
heads on the ceiling. That's why the crew tells you to buckle your seatbelts. Normally we assume that when an
object is at rest, it will stay at rest unless a force acts on it. Furthermore, Galileo realized that
an object in motion will tend
to stay in motion. This tendency to keep on doing the same thing is called \emph{inertia}.
One of our cues that the bouncing cabin is not a valid frame of reference is that if we use the cabin as a
reference point, objects do not appear to behave inertially. The person who goes flying up and hits the ceiling
was not acted on by any force, and yet he changed from being at rest to being in motion. We would interpret this
as meaning that the plane's motion was not steady. The person didn't really pop up and hit the ceiling. What really
happened, as suggested in the figure, was that the ceiling swerved downward and hit the person on the head. The cabin is a \emph{noninertial frame
of reference}. Moving frames are all right, but noninertial ones aren't.
