\chapter{Galilean relativity}

\section{Galileo versus Aristotle}

Once upon a time, there was an ornery man who liked to argue.
He was undiplomatic and had a talent for converting allies into enemies.
His name was Galileo, and it was his singular misfortune to be correct in most of his opinions.
When he was arguing, Galileo had a few annoying habits. One was to answer a perfectly sound theoretical
argument with a contradictory experiment or observation. Another was that, since he
knew he was right, he freely made up descriptions of experiments that he hadn't actually done.

Galileo's true opponent was a dead man, the ancient Greek philosopher Aristotle.
Aristotle, probably based on generalizations from everyday experience, had come up
with some seemingly common-sense theories about motion.

\pagebreak

The figure \vpageref{fig:soccer-ball} shows an example of the kind of observation that might lead you to
the same conclusions as Aristotle. It shows a series of snapshots in the motion
of a rolling ball. Time moves forward as we go up the page.
Because the ball slows down and eventually stops, it traces out the shape of a
letter ``J.'' If you get rid of the artistic details, and connect the drawings of
the ball with a smooth curve, you get a graph, with position and time on the
horizontal and vertical axes. (If we had wanted to, we could have interchanged
the axes or reversed either or both of them. Making the time axis vertical, and
making the top point toward the future, is a standard convention in relativity,
like the conventional orientation of the compass directions on a map.)

Once the ball has stopped completely, the graph becomes a vertical line. Time
continues to flow, but the ball's position is no longer changing. According to
Aristotle, this is the natural behavior of any material object. The ball may move
because someone kicks it, but once the force stops, the motion goes away.
In other words, vertical lines on these graphs are special. They represent a natural, universal
state of rest.

\begin{figure}\label{fig:soccer-ball}
\fignakedfullpage{soccer-ball}
\end{figure}

Galileo adduced two main arguments against the Aristotelian view. First, he said
that there was a type of force, friction, which was the reason that things slowed down.
The grass makes a frictional force on the ball. If we let the grass grow too high,
this frictional force gets bigger, and the ball decelerates more quickly. If we mow
the grass, the opposite happens. Galilo did experiments in which he rolled a smooth
brass ball on a smooth ramp, in order to make the frictional force as small as possible.
He carried out careful quantitative observations of the balls' motion, with time
measured using a primitive water clock. By taking measurements under a variety of
conditions, he showed by extrapolation from his data that if the ramp was perfectly level,
and friction completely absent, the ball would roll forever. 

Today we have easier ways
to convince ourselves of the same conclusion. For example, you've seen a puck glide frictionlessly on an air hockey table
without slowing down.

Galileo's second argument had to do with the fact that we can't judge motion except by
comparing with some reference point, which we consider to be stationary. Aristotle
used the earth as a reference point. 
But in 1610, Galileo looked at the planet Jupiter through a telescope that he had invented, and
saw something startling. The planet, which looks like a bright and starlike point to the naked eye,
appeared as a disk, and accompanying it were four points of light. Observing them from one
night to the next, as shown in the notebook page reproduced here, he saw that they wove their
way back and forth, to the east and west of Jupiter's disk, in the plane of the solar system.
He inferred that they were moons
that were circling Jupiter, and that he was seeing the circles from the side, so
that they appeared flat. The figure \vpageref{fig:galileo-jovian-moons-sketch} shows the original diagrams from Galileo's notebook.
(The analogy with our diagram of the rolling ball is nearly exact,
except that Galileo worked his way down the page from night to night, rather than going up.)

The behavior of these moons is nothing like what Aristotle would have predicted. The earth
plays no special role here. The moons circle Jupiter, not the earth. Galileo's observations demoted
the earth from its Aristotelian place as the universal reference point for the rest of the cosmos,
making it into just one of several planets in the solar system. (It wasn't a coincidence that Galileo
was also a strong supporter of Copernicus's theory that the sun rather than the earth was the center
around which the planets circled.)

\begin{figure}\label{fig:galileo-jovian-moons-sketch}
\fignakedfullpage{galileo-jovian-moons-sketch}
\end{figure}

\vfill\pagebreak

\section{Frames of reference}

If Aristotle had grown up on Jupiter, maybe he would have considered Jupiter to be the natural reference
point that was obviously at rest. In fact there is no one reference point that is always right. Psychologically,
we have a strong tendency to think of our immediate surroundings as being at rest. For example, when you fly
on an airplane, you tend to forget that the cabin is hurtling through the sky at nearly the speed of sound.
You adopt the \emph{frame of reference} of the cabin.
If you were to describe the motion of the drink cart progressing down the aisle with your nuts and soda,
you might say, ``Five minutes ago, it was at row 23, and now it's at row 38.'' You've implicitly laid out
a number line along the length of the plane, with a reference point near the front hatch (where ``row zero'' would be).
The measurements like 23 and 38 are called \emph{coordinates}, and in order to define them you need to pick
a frame of reference.

\begin{figure}\label{fig:cow-and-car}
\fignakedfullpage{cow-and-car}
\end{figure}

In the top part of the figure \vpageref{fig:cow-and-car}, we adopt the cow's frame of reference. The cow sees itself as being at rest,
while the car drives by on the road. The version at the bottom shows the same situation in
the frame of the driver,
who considers herself to be at rest while the scenery rolls by. To the driver, it's the cow, the grass, and
the trees that are moving.

Aristotle believed that there was one special, or ``\emph{preferred}'' frame of reference, which was
the one attached to the earth. One of the basic principles of relativity is that there is no such
preferred frame.

\vfill\pagebreak

\section{Inertial and noninertial frames}

That's not to say that all frames of reference are equally good. It's a little like the bitter joke
in George Orwell's satirical novel Animal Farm: ``All animals are equal, but some animals are more equal than others.''
For an example of the distinction between good and bad frames of reference, consider what it feels like on
an airplane when you hit some turbulence. You can tell that you're getting bumped around, it's not just
a matter of opinion whether anything special is happening, and you can tell this without reference to anything
external. Similarly, when a movie director wants to depict an earthquake, or the scene aboard a ship that's
been hit by a torpedo, the low-budget technique is to shake the camera around --- but we can tell it's fake
(unless the actors do a really good job of pretending to fall down or clutch the set). The shaking camera is
not a valid frame.

When a plane hits bad turbulence, it's possible for passengers to go popping out of their seats and hit their
heads on the ceiling. That's why the crew tells you to buckle your seatbelts. Normally we assume that when an
object is at rest, it will stay at rest unless a force acts on it. Furthermore, Galileo realized that
an object in motion will tend
to stay in motion. This tendency to keep on doing the same thing is called \emph{inertia}.
One of our cues that the bouncing cabin is not a valid frame of reference is that if we use the cabin as a
reference point, objects do not appear to behave inertially. The person who goes flying up and hits the ceiling
was not acted on by any force, and yet he changed from being at rest to being in motion. We would interpret this
as meaning that the plane's motion was not steady. The person didn't really pop up and hit the ceiling. What really
happened was that the ceiling swerved downward and hit the person on the head. The cabin is a \emph{noninertial frame
of reference}. Moving frames are all right, but noninertial ones aren't.

There is a subtle chicken-and-egg problem, however, with defining a distinction between inertial and noninertial
frames of reference. In the figure \vpageref{fig:cow-and-car}, suppose we're already convinced that the cow's
frame is inertial. The cow sees the car moving along in a straight line, at a speed that appears to be constant
in the cow's frame. Therefore the cow says that the car's motion is inertial --- the car isn't going over bumps,
running into a tree, or anything like that. The cow therefore infers that a frame of reference tied to the car
would also be a valid inertial frame.

Similarly, if the driver is already convinced that the car's motion is inertial, then she can tell that the cow,
who to her is moving backward at constant speed, also has a good inertial frame.

But now perhaps you've noticed the difficulty. Given one inertial frame of reference, we can tell which other
frames are also inertial. But how do we get started? Similarly, imagine that you were in Tanzania,
and needed to make sure that the money you were handling was valid currency. If you had a 1000 shilling note from
a reputable source, you could compare other bills with it and make sure they looked the same. But without a
known-good example to start with, you'd have no way of telling whether shopkeepers were all handing you Monopoly money
as change.

Resolving this issue a little tricky. Among the people who tried have been
Galileo (1564-1642), his successor Isaac Newton (1642-1726), and
Albert Einstein (1879-1955). None of them really got it quite right by modern standards, and the reason has
to do with gravity.

\vfill\pagebreak

\section{Gravity}

The main problem is in pinning down a definition of whether an object is moving inertially. We have in mind
some kind of situation in which no forces act on it. But how do we know whether this is the case?
It turns out to be easy for most types of forces, but not for gravity.

The first figure \vpageref{fig:rock-on-string-and-earth} shows an overhead view of a person whirling a rock in a circle on the end
of a string. We have a number of cues to tell us  that there is a force acting on the rock.
The string is taut, it's tied around the rock, and so on. If the string breaks, the force stops acting,
and the rock's motion becomes inertial. It flies off in a straight line at constant speed.

\begin{figure}\label{fig:rock-on-string-and-earth}
\fignakedfullpage{rock-on-string-and-earth}
\end{figure}

But what if the object going in a circle is the earth orbiting the sun? We've been told in school
that the force acting on it is the sun's gravity, but what signs do we have that this force is acting?
Unlike a hypothetical bug riding on the circling rock, we riders on the earth get no physical sensation
to tell us that this force exists at all.
If it stopped acting, we wouldn't feel any immediate sign of its cessation.
As suggested by the figure, we would just eventually notice that we seemed to be receding from the sun.

The reason we don't sense the sun's gravitational force is that it acts simultaneously and with
equal effect on both the earth and our bodies. Perhaps a simpler example is an amusement park ride near
where I live, called the Supreme Scream. Riders are lifted in chairs up to the top of a tower and then
dropped in free-fall straight down toward the ground. A popular thing to do on this ride is to get a coin ready at the
top, and then release it once the free fall has begun. It  hangs in the air in front of you, apparently
weightless, although both you and it are actually accelerating downward at the same rate. Once you get
off the ride, the change you feel is not any sudden reappearance of the earth's gravity, which was present all
along. It's the resumption of the usual force of the ground pressing up on your feet.

In this sense, gravity is different from the other forces of nature, such as magnetic or nuclear forces.
We can't shield against it, and we can't necessarily tell whether it's even acting. This is an example
of something called the \emph{equivalence principle}, which we'll come back to later in our study of relativity.

\begin{figure}\label{fig:waage-box}
\fignakedfullpage{waage-box}
\end{figure}

\section{The law of inertia}

For these reasons, we state a \emph{law of inertia} in a way that sidesteps the question of gravity.
We imagine building a box with layers of shielding, as suggested in the cutaway view in the fanciful figure
\vpageref{fig:waage-box},
where the box is floating in outer space.
One layer of shielding is aluminum foil to keep out any external electric forces. Another layer provides
shielding against magnetism. (There are special products sold for this purpose, with trade names such
as mu-metal.) We shield against all the forces except for gravity, which, as far as we know, cannot be
shielded against. Now inside the box we introduce a black ball and a white ball. The roles played by the
two balls are completely symmetrical, but if we like, we can think of one ball as playing the role of
an observer. For this reason I've drawn a ladybug clinging to the white ball. The bug can, if she
likes, define herself to be at rest, so that in her frame of reference only the black ball is moving.

The law of inertia states that if all nongravitational forces are prevented from acting on the balls,
then each will move in a straight line, at constant speed, relative to the other ball. This type of
straight-line, constant-speed motion is called inertial motion. To the ladybug,
the black ball's motion appears inertial.

Although I drew the box drifting in the interstellar void, nothing goes wrong if we apply this definition
with the box resting on a tabletop on earth. In this case, the two balls will be affected equally by the earth's
gravity, and each will still move inertially relative to the other.
Nor is such a box completely impractical to build;
Harold Waage at Princeton has made something similar to it as a lecture demonstration.

If you're accustomed to dictionary-style definitions, or the kind of definitions used in mathematics,
it may seem strange to define something using a piece of apparatus.
This is in fact a very common and philosophically well justified approach called operationalism.
The idea is that if we want to define something,
we should simply spell out the operations needed in order to measure or verify it.
For example, the operational definition of time would be that time is what is measured by a clock.

Implicit in this description is the fact that no frame of reference is preferred. We could have perched the
bug on the black ball, and that would have been an equally valid frame.
