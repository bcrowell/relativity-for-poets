\chapter{Galilean relativity}\label{ch:galilean-relativity}

\section{Galileo versus Aristotle}

Once upon a time, there was an ornery man who liked to argue.
He was undiplomatic and had a talent for converting allies into enemies.
His name was Galileo, and it was his singular misfortune to be correct in most of his opinions.
When he was arguing, Galileo had a few annoying habits. One was to answer a perfectly sound theoretical
argument with a contradictory experiment or observation. Another was that, since he
knew he was right, he freely made up descriptions of experiments that he hadn't actually done.

Galileo's true opponent was a dead man, the ancient Greek philosopher Aristotle.
Aristotle, probably based on generalizations from everyday experience, had come up
with some seemingly common-sense theories about motion.

\pagebreak

The figure \vpageref{fig:soccer-ball} shows an example of the kind of observation that might lead you to
the same conclusions as Aristotle. It shows a series of snapshots in the motion
of a rolling ball. Time moves forward as we go up the page.
Because the ball slows down and eventually stops, it traces out the shape of a
letter ``J.'' If you get rid of the artistic details, and connect the drawings of
the ball with a smooth curve, you get a graph, with position and time on the
horizontal and vertical axes. (If we had wanted to, we could have interchanged
the axes or reversed either or both of them. Making the time axis vertical, and
making the top point toward the future, is a standard convention in relativity,
like the conventional orientation of the compass directions on a map.)

Once the ball has stopped completely, the graph becomes a vertical line. Time
continues to flow, but the ball's position is no longer changing. According to
Aristotle, this is the natural behavior of any material object. The ball may move
because someone kicks it, but once the force stops, the motion goes away.
In other words, vertical lines on these graphs are special. They represent a natural, universal
state of rest.

\begin{figure}\label{fig:soccer-ball}
\fignakedfullpage{soccer-ball}
\end{figure}

Galileo adduced two main arguments against the Aristotelian view. First, he said
that there was a type of force, friction, which was the reason that things slowed down.
The grass makes a frictional force on the ball. If we let the grass grow too high,
this frictional force gets bigger, and the ball decelerates more quickly. If we mow
the grass, the opposite happens. Galilo did experiments in which he rolled a smooth
brass ball on a smooth ramp, in order to make the frictional force as small as possible.
He carried out careful quantitative observations of the balls' motion, with time
measured using a primitive water clock. By taking measurements under a variety of
conditions, he showed by extrapolation from his data that if the ramp was perfectly level,
and friction completely absent, the ball would roll forever. 

Today we have easier ways
to convince ourselves of the same conclusion. For example, you've seen a puck glide frictionlessly
across an air hockey table without slowing down.
Sometimes Galileo's analysis may be hard to accept. Running takes us a lot of effort, and
it seems as though we need to apply a continuous force to keep going. The figure above shows what's really
happening. At b, the runner's heel strikes the ground, and friction slows him down. A forward
force in c serves only to recover from the loss in speed.

\begin{figure}\label{fig:runner}
\fignakedfullpage{runner}
\end{figure}

Galileo's second argument had to do with the fact that we can't judge motion except by
comparing with some reference point, which we consider to be stationary. Aristotle
used the earth as a reference point. 
But in 1610, Galileo looked at the planet Jupiter through a telescope that he had invented, and
saw something startling. The planet, which looks like a bright and starlike point to the naked eye,
appeared as a disk, and accompanying it were four points of light. Observing them from one
night to the next, as shown in the notebook page reproduced here, he saw that they wove their
way back and forth, to the east and west of Jupiter's disk, in the plane of the solar system.
He inferred that they were moons
that were circling Jupiter, and that he was seeing the circles from the side, so
that they appeared flat. The figure \vpageref{fig:galileo-jovian-moons-sketch} shows the original diagrams from Galileo's notebook.
(The analogy with our diagram of the rolling ball is nearly exact,
except that Galileo worked his way down the page from night to night, rather than going up.)

The behavior of these moons is nothing like what Aristotle would have predicted. The earth
plays no special role here. The moons circle Jupiter, not the earth. Galileo's observations demoted
the earth from its Aristotelian place as the universal reference point for the rest of the cosmos,
making it into just one of several planets in the solar system.

\begin{figure}\label{fig:galileo-jovian-moons-sketch}
\fignakedfullpage{galileo-jovian-moons-sketch}
\end{figure}

\vfill\pagebreak

\section{Frames of reference}\label{sec:frames-of-reference}

If Aristotle had grown up on Jupiter, maybe he would have considered Jupiter to be the natural reference
point that was obviously at rest. In fact there is no one reference point that is always right. Psychologically,
we have a strong tendency to think of our immediate surroundings as being at rest. For example, when you fly
on an airplane, you tend to forget that the cabin is hurtling through the sky at nearly the speed of sound.
You adopt the \emph{frame of reference} of the cabin.
If you were to describe the motion of the drink cart progressing down the aisle with your nuts and soda,
you might say, ``Five minutes ago, it was at row 23, and now it's at row 38.'' You've implicitly laid out
a number line along the length of the plane, with a reference point near the front hatch (where ``row zero'' would be).
The measurements like 23 and 38 are called \emph{coordinates}, and in order to define them you need to pick
a frame of reference.

\begin{figure}\label{fig:cow-and-car}
\fignakedfullpage{cow-and-car}
\end{figure}

In the top part of the figure \vpageref{fig:cow-and-car}, we adopt the cow's frame of reference. The cow sees itself as being at rest,
while the car drives by on the road. The version at the bottom shows the same situation in
the frame of the driver,
who considers herself to be at rest while the scenery rolls by.

Aristotle believed that there was one special, or \emph{preferred} frame of reference, which was
the one attached to the earth. Relativity says there is no such
preferred frame.

\section{Inertial and noninertial frames}

That's not to say that all frames of reference are equally good. It's a little like the line from
Orwell's satirical novel Animal Farm: ``All animals are equal, but some animals are more equal than others.''
For an example of the distinction between good and bad frames of reference, consider a low-budget movie in which
the director wants to depict an earthquake.
The easy way to do it is to shake the camera around --- but we can tell it's fake.

\begin{figure}[h]\label{fig:plane-noninertial}
\fignakedfullpage{plane-noninertial}
\end{figure}

Similarly, when a plane hits bad turbulence, it's possible for passengers to go popping out of their seats and hit their
heads on the ceiling. That's why the crew tells you to buckle your seatbelts. Normally we assume that when an
object is at rest, it will stay at rest unless a force acts on it. Furthermore, Galileo realized that
an object in motion will tend
to stay in motion. This tendency to keep on doing the same thing is called \emph{inertia}.
One of our cues that the bouncing cabin is not a valid frame of reference is that if we use the cabin as a
reference point, objects do not appear to behave inertially. The person who goes flying up and hits the ceiling
was not acted on by any force, and yet he changed from being at rest to being in motion. We would interpret this
as meaning that the plane's motion was not steady. The person didn't really pop up and hit the ceiling. What really
happened, as suggested in the figure, was that the ceiling swerved downward and hit the person on the head. The cabin is a \emph{noninertial frame
of reference}. Moving frames are all right, but noninertial ones aren't.

\section{Addition of velocities}\label{sec:galilean-velocity-addition}

All inertial frames of reference are equally valid. However, observers in different frames of reference
can give different answers to questions and get different, correct answers from measurements. For example,
suppose that Sparkly Elf Lady is riding her unicorn at 30 kilometers per hour to the east, and she fires
an arrow from her bow. To her, the bow fires the arrow at its normal speed of 20 km/hr. To an observer
watching her go by, however, the arrow is going 50 km/hr. That is, to convert velocities between different
frames of reference, we add and subtract.

\section{The Galilean twin paradox}\label{sec:galilean-twin-paradox}

Alice and Betty are identical teenage twins. Betty goes on a space voyage to get a fish taco with
cilantro and grated carrots from a
taco shack that has the best food in the known universe. Alice stays home. The diagram shows the story using our usual
graphical conventions, with time running vertically and space horizontally.

\begin{figure}[h]\label{fig:galilean-twin-paradox}
\fignakedfullpage{galilean-twin-paradox}
\end{figure}

Motion is relative, so it seems that it should be equally valid to consider Betty
and the spaceship as having been at rest the whole time, while
Alice and the planet earth traveled away from the spaceship along
line segment 3 and then returned via 4. But this is not consistent with the
experimental results, which show that Betty undergoes a violent
deceleration and reacceleration at her turnaround point, while Alice and the other
inhabitants of the earth feel no such effect.

The paradox is resolved because Galilean relativity doesn't say that \emph{all}
motion is relative. Only inertial motion is relative. When motion is noninertial, we can get detectable
physical effects. The earth's motion is inertial, and the spaceship's isn't.


\pagebreak

\section{The Galilean transformation}

\begin{figure}\label{fig:galilean-transformation}
\fignakedfullpage{galilean-transformation}
\end{figure}

In section \ref{sec:frames-of-reference} on p.~\pageref{sec:frames-of-reference}, we visualized the frames of reference
of the cow and the car by making two different diagrams of the motion, one with the cow at rest and one with the car
at rest. Rather than making two separate graphs, we can be more economical by drawing a single diagram,
but superimposing two different graph-paper grids, as shown above.

The white grid is what the cow would use.
The car travels one meter
forward for every second of time.

The driver of the car would prefer the black grid. According to this grid, the cow starts at $t=0$
with a position of 3 meters, and by $t=3$ seconds, it's gone backward to a position of 0 meters.

What we're doing here is switching back and forth between the coordinates of two different frames
of reference. Coordinates are like names that we attach to events. When we switch coordinate systems,
reality doesn't change, only the names do. It's just like translating words to a different language.
This method of translating between two frames of reference is called the Galilean transformation.

\pagebreak

\sectionwithsurtitle{Science and society}{The Galileo affair}{The Galileo affair}

Galileo had a famous showdown with the Church, which ought to teach us something useful about
the relationship between science and religion and their roles in society. But having taught
about this topic for twenty years, I find that the more I learn about it, the less sure I am
of what lessons to draw.

First some background. Galileo's lifetime (1564-1642) coincided exactly
with the Counter-Reformation, and during his mature years Europe was consumed by a period
of brutal warfare much like what we see today in places like Iraq, with religious and state
powers cynically using one another to further their own agendas. In some areas, 75\% of
the population was killed by the war and its side-effects, including disease and famine.
In parallel with the hard work of physical slaughter, there was an intellectual battle going on.
New religious orders such as the Theatines were founded, with the mission of defending Italy from
the Protestant heresy. In 1559 the Church published an
Index of Forbidden Books (abolished in 1966), and in 1588 the Roman Inquisition was established.
During the height of the conflict Galileo lived in relative safety near Florence, with a Medici prince as his patron.
Adding to the apparent security of his position, his scientific work was approved of by both
the Jesuits and his friend Cardinal Barberini, who in 1623 became Pope Urban VIII.

But Galileo was a flamboyant personality and a best-selling author, and he ended up
getting in trouble by very publicly advocating the Copernican system of the universe,
in which the planets, including the earth, circled the sun. After some initial ignorant
fulmination against Galileo from the pulpit, the first real shot across the
bow came in the form of an intellectually sophisticated 1616 letter to Galileo from the Theatine priest Ingoli,
who had participated in the Accademia dei Lincei, of which Galileo was a leading member.
Apologists for the Church love Ingoli's letter, because it zeroes in on two
real scientific weaknesses in Galileo's position. This is contrary to the usual
picture of the Church persecuting poor Galileo purely because they thought a moving
earth contradicted passages from the Bible, e.g., ``Tremble before him, all the earth.
The world is established and can never be moved,'' (1 Chronicles 16:30). Although
Ingoli does include such theological points, they are downplayed.
Ingoli, basing his main arguments on work by the astronomer Tycho Brahe, points out
two purely \emph{scientific} problems with Copernicanism:

\begin{enumerate}
\item Based on the best available data, the scale
of a Copernican universe would have had to be such that the stars were implausibly large (about the
size of our entire solar system).
\item If the earth were spinning on its own axis, then there would be detectable effects on
the motion of projectiles, but no such effects are observed.
\end{enumerate}

The first argument turns out to have been based on mistaken data, but we can recognize the
second as an argument about \emph{relativity}.  Let's analyze this with the benefit of
modern hindsight, since neither Galileo nor Ingoli had a clear enough concept of
inertial motion to be able to attack it definitively.

One's initial reaction might be that motion is relative,
so if we see the sun appear to spin around the earth, isn't it just
a matter of opinion whether the earth is spinning or the sun revolving around it? From
this point of view, arguing about an earth-centered cosmos versus a sun-centered one is
like arguing about whether it's really the cow or the car on p.~\pageref{fig:cow-and-car} that moves.
This was in fact essentially the position taken by the higher Church officials later on: they
told Galileo that he could discuss Copernicanism as a mathematical hypothesis, but not as
a matter of physical fact.

But this argument is just plain wrong, since inertial motion is not just motion at constant
speed, it's motion that is also in a \emph{straight line}. Therefore motion in a circle really
\emph{is} more than a matter of opinion, and in fact we \emph{can} detect the effects of
the earth's rotation on projectiles. For example, when powerful naval guns fire a shell to the
north, in the northern hemisphere, the shell can fly as far as 30 km, and its motion carries it
to a higher latitude, at which the earth's rotation has a smaller velocity. (As an extreme
case, Santa's candy-cane pole doesn't go anywhere as the earth rotates, it just spins in place.)
The projectile retains its original, higher eastward velocity, so to an observer on the earth's
surface, it appears to swerve to the east. This is known as the Coriolis effect, and is also
the explanation for the rotation of cyclones. So Galileo was right about the earth's spin, but
for the wrong reasons. At the time, the state of physics simply wasn't advanced enough to
accurately calculate the effect Ingoli described. Such a calculation would have shown that the
effect was too small to have been observable with contemporary technology.

The aged Galileo persisted stubbornly in advocating Copernicanism, and in
1633 he was sentenced to house arrest for the rest of his life. Tradition holds that after the
sentence was pronounced, he muttered, ``And yet, it does move.''

So what do we learn from all this? A fairly common interpretation is that the whole conflict
was needless. There doesn't have to be any conflict between science and religion. We simply need
to keep science, religion, and the state within their separate and proper spheres of authority.
Although I've given this interpretation to my students in the past, I can't help feeling now
that I was being a Pollyanna. As I write this, the press is reporting on the killing of 12
journalists and 4 Jews by Muslim fundamentalists in Paris, as well as the massacre of
2,000 civilians in northern Nigeria by Boko Haram, an Islamic fundamentalist group whose
name means literally ``books are a sin.'' Maybe a more realistic moral to draw from
the Galileo story is that religion is an inherently dangerous force in the world, one which
tends to wreak havoc in the secular world unless it's backed into a corner with a whip,
like a snarling circus lion.
