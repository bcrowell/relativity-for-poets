\chapter{Einstein's relativity}

\section{Time is relative}

In chapter \ref{ch:galilean-relativity} I laid out a beautiful theory of motion, which,
by the way, is wrong. To see why it's wrong, we have to think about the assumptions hidden
within it about space and time.

Suppose that I drive to
Gettysburg, Pennsylvania, and stand in front of the brass plaque that marks the site of
the momentous Civil War battle. There I am, at the same place where it all happened.
But wait, am I really in the same place? An observer whose frame of
reference was fixed to another planet would say that our planet had moved through space since 1863.

\vfill\pagebreak

Consider the figure \vpageref{fig:no-simulplaceity}. The top diagram shows the inertial motion of the earth
as you'd learn to represent it if Aristotle was your schoolteacher. He tells you to use lined notebook
paper, because its blue parallel lines represent a state of rest. The series of positions of the earth
has to be drawn so that it lines up correctly. The earth is, after all, at rest.

\begin{figure}\label{fig:no-simulplaceity}
\fignakedfullpage{no-simulplaceity}
\end{figure}

But then your parents pull you out of the strict, structured Aristotelian Academy
and enroll you in the hippy-dippy,
color-outside-the-lines Galilean School for Groovy Boys and Girls.
At the Galilean School, the teachers just hand out blank pieces of white typing paper.
Your new teachers tell you it's all right to make your diagram however you like, as
long as the earth's positions form a line.

In other words, the things we've been saying about motion can also be taken as claims about
the structure of the background, the stage of space and time on which motion is played out.
We call this background \emph{spacetime}, and each point on it is an \emph{event}: a certain time and
place, such as July 1, 1863 at Gettysburg, Pennsylvania.

Aristotelian spacetime comes equipped with a special structure, represented by the blue
lines on the notebook paper. They represent a preferred frame of reference, and they always
allow us to decide unambiguously whether or not two events happened at the \emph{same place}.
Galileo corrected this view by showing that spacetime didn't actually have any such structure.
We can't say in any absolute sense whether two events happen at the same place. It's a matter
of opinion, because it depends on the frame of reference we pick.

But \emph{both} Galileo and Aristotle implicitly assumed a further type of structure
for spacetime, which we can visualize as a set of \emph{horizontal} lines on the paper.
Events lying on the same horizontal line are \emph{simultaneous}, and according to them
simultaneity is not a matter of opinion. It's absolute. Neither Galileo nor Aristotle realized
that this was a nontrivial assumption, but Newton did, and he explicitly stated it:
%
\begin{quote}
Absolute, true, and mathematical time, of itself, and from its own nature,
flows uniformly without relation to anything external [...]
\end{quote}
%
The figure \vpageref{fig:hk-in-cabin} shows an experiment that proved this assumption to be wrong.

\begin{figure}\label{fig:hk-in-cabin}
\fignakedfullpage{hk-in-cabin}
\end{figure}
