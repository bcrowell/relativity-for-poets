\chapter{Einstein's relativity}

\section{Time is relative}

In chapter \ref{ch:galilean-relativity} I laid out a beautiful theory of motion, which,
by the way, is wrong. To see why it's wrong, we have to think about the assumptions hidden
within it about space and time.

Suppose that I drive to
Gettysburg, Pennsylvania, and stand in front of the brass plaque that marks the site of
the momentous Civil War battle. There I am, at the same place where it all happened.
But wait, am I really in the same place? An observer whose frame of
reference was fixed to another planet would say that our planet had moved through space since 1863.

\vfill\pagebreak

Consider the figure \vpageref{fig:no-simulplaceity}. The top diagram shows the inertial motion of the earth
as you'd learn to represent it if Aristotle was your schoolteacher. He tells you to use lined notebook
paper, because its blue parallel lines represent a state of rest. The series of positions of the earth
has to be drawn so that it lines up correctly. The earth is, after all, at rest.

\begin{figure}\label{fig:no-simulplaceity}
\fignakedfullpage{no-simulplaceity}
\end{figure}

But then your parents pull you out of the strict, structured Aristotelian Academy
and enroll you in the hippie-dippy,
color-outside-the-lines Galilean School for Groovy Boys and Girls.
At the Galilean School, the teachers just hand out blank pieces of white typing paper.
Your new teachers tell you it's all right to make your diagram however you like, as
long as the earth's positions form a line.

In other words, the things we've been saying about motion can also be taken as claims about
the structure of the background, the stage of space and time on which motion is played out.
We call this background \emph{spacetime}, and each point on it is an \emph{event}: a certain time and
place, such as July 1, 1863 at Gettysburg, Pennsylvania.

Aristotelian spacetime comes equipped with a special structure, represented by the blue
lines on the notebook paper. They represent a preferred frame of reference, and they always
allow us to decide unambiguously whether or not two events happened at the \emph{same place}.
Galileo corrected this view by showing that spacetime didn't actually have any such structure.
We can't say in any absolute sense whether two events happen at the same place. It's a matter
of opinion, because it depends on the frame of reference we pick.

But \emph{both} Galileo and Aristotle implicitly assumed a further type of structure
for spacetime, which we can visualize as a set of \emph{horizontal} lines on the paper.
Events lying on the same horizontal line are \emph{simultaneous}, and the assumption has been that
simultaneity is not a matter of opinion. It's absolute.

\pagebreak

Neither Galileo nor Aristotle realized
that this was a nontrivial assumption, but Isaac Newton (1642-1726) did, and he explicitly stated it:
%
\begin{quote}
Absolute, true, and mathematical time, of itself, and from its own nature,
flows uniformly without relation to anything external [...]
\end{quote}

\begin{figure}[h]\label{fig:hk-in-cabin}
\fignakedfullpage{hk-in-cabin}
\end{figure}

The photo
% \vpageref{fig:hk-in-cabin}
shows a 1971 experiment that proves this hidden assumption to be wrong.
There had been plenty of earlier evidence, but this experiment is particularly direct and easy to explain.
Physicist J.C.~Hafele and astronomer R.E.~Keating  brought atomic clocks aboard commercial
airliners and flew around the world, once from east to west and once from west to east.
The clocks took up two seats, and two tickets were bought for them under the name of ``Mr.~Clock.''

When they got back, Hafele and Keating observed that there was a discrepancy between the times measured by the
traveling clocks and the times measured by similar clocks that stayed home at the U.S.~Naval Observatory in Washington.
The effect was small enough that it has to be expressed in units of nanoseconds (ns), or billionths of a second.
The east-going clock lost time, ending up off by $-59\ \zu{ns}$, while the west-going 
one gained $273\ \zu{ns}$. Although the effects were small, they were statistically significant compared
to the clocks' margins of error of about $\pm 10\ \zu{ns}$. 

Actually it makes sense that the effects were small.
Galileo's description of spacetime  had already been thoroughly tested by experiments under a wide variety of conditions,
so even if it's wrong, and we're going to replace it with some new theory, the
new theory must agree with Galileo's to a good approximation, within the Galilean theory's
realm of applicability. This requirement of backward-compatibility is known as
the \emph{correspondence principle}.

It's also reassuring that the effects on time were
small compared to the three-day lengths of the plane trips. There was therefore no
opportunity for paradoxical scenarios such as one in which the east-going experimenter arrived
back in Washington before he left and then convinced himself not to take the trip.
A theory that maintains this kind of orderly relationship between cause and effect is said to satisfy causality.

Hafele and Keating were not working in the dark. They already had access to an improved
theory of spacetime, which was Albert Einstein's theory of relativity, developed around 1905-1916.
They were testing relativity's predictions (and they were not the first to do so).
Let's work backward instead, and
inspect the empirical results for clues as to how time works.
The east-going clock lost time, while the west-going one gained.
Since two traveling clocks experienced effects in opposite directions,
we can tell that the rate at which time flows depends on the motion
of the observer. The east-going clock was moving in the same direction as the earth's rotation, so its velocity
relative to the earth's center was greater than that of the clock that remained in Washington, while the west-going 
clock's velocity
correspondingly reduced. The fact that the east-going clock fell behind, and the west-going one got ahead,
shows that the effect of motion is to make time go more slowly. This effect of motion on time was predicted by
Einstein in his original 1905 paper on relativity, written when he was 26.

\begin{figure}[h]\label{fig:iijima}
\fignakedfullpage{iijima}
\end{figure}

If this had been the only effect in the Hafele-Keating experiment, then we would have expected to see effects on the
east-going and west-going clocks that were equal in size.
In fact, the two effects are unequal. This
implies that there is a second effect involved, simply due to the planes' being up in the air.
This was verified more directly in a 1978 experiment by Iijima and Fujiwara, in which one clock was
kept in Mitaka, a suburb of Tokyo, while an identical clock was driven up nearby Mount Norikura,
left motionless there for about a week, and then brought back down and compared with the one at the bottom of the mountain.
Their results are shown in the graph.
This experiment, unlike the Hafele-Keating one, isolates one effect on time, the gravitational one: time's
rate of flow increases with height in a gravitational field. Einstein didn't figure out
how to incorporate gravity into relativity until 1915, after much frustration and many false starts. The
simpler version of the theory without gravity is known as special relativity, the full version as general
relativity. We'll restrict ourselves to special relativity for now, and that means that what we want to
focus on right now is the distortion of time due to motion, not gravity.


By the way, the effects described in these atomic clock experiments could have seemed obscure
to paypeople in the 1970s, but today they are part of everyday life, because the GPS system depends
crucially on them. A GPS satellite in orbit experiences effects due to motion and gravity that are both
much larger than the corresponding effects in the Hafele-Keating and Iijima experiments. The satellites carry
atomic clocks, and beam down time-stamped
radio signals to the earth to  receivers such as the ones used by motorists and hikers.
By comparing the time stamps of signals from several different satellites, the receiver can calculate
how long the signals took to travel to it at the speed of light, and therefore determine its own position.
The GPS network started out as a US military system, and this was why Hafele and Keating were able to get
funding from the Navy. There is a legend that the military brass in charge of the program weren't quite sure
they believed in all the crazy relativity stuff being spouted by the longhaired physics professors, so they
demanded that the satellites have special software switches designed into them so that the relativity corrections
could be turned off if necessary. In fact, both special and general relativity are crucially important to GPS,
and the system would be completely broken without them.

\begin{figure}\label{fig:correspondence-dramatized}
\fignakedfullpage{correspondence-dramatized}
\end{figure}

We can now see in more detail how to apply the correspondence principle. The behavior of the clocks in the
Hafele-Keating experiment shows that the amount of time distortion increases as the speed of the clock's motion
increases. Galileo lived in an era when the fastest
mode of transportation was a galloping horse, and the best
pendulum clocks would accumulate errors of perhaps a minute over the course of several days.
A horse is much slower than a jet plane, so the
distortion of time would have had a relative size of only one part in $1,000,000,000,000,000$ ($10^{15}$)
--- much smaller than the clocks were capable of detecting.
At the speed of a passenger jet, the effect is about one part in 1,000,000,000,000 ($10^{12}$),
and state-of-the-art atomic clocks in 1971 were capable of measuring that.
A GPS satellite travels much faster than a jet airplane, and the effect on the satellite
turns out to be about one part in 10 billion ($10^{10})$. The general idea here is that all physical laws are approximations, and
approximations aren't simply right or wrong in different situations. Approximations are better or worse
in different situations, and the question is whether a particular approximation is good enough in a given
situation to serve a particular purpose. The faster the motion, the worse the Galilean approximation of
absolute time. Whether the approximation is good enough depends on what you're trying to accomplish.
The correspondence principle says that the approximation must have been good enough to explain
all the experiments done in the centuries before Einstein came up with relativity.
